\documentclass{beamer}\usepackage[]{graphicx}\usepackage[]{color}
%% maxwidth is the original width if it is less than linewidth
%% otherwise use linewidth (to make sure the graphics do not exceed the margin)
\makeatletter
\def\maxwidth{ %
  \ifdim\Gin@nat@width>\linewidth
    \linewidth
  \else
    \Gin@nat@width
  \fi
}
\makeatother

\definecolor{fgcolor}{rgb}{0.345, 0.345, 0.345}
\newcommand{\hlnum}[1]{\textcolor[rgb]{0.686,0.059,0.569}{#1}}%
\newcommand{\hlstr}[1]{\textcolor[rgb]{0.192,0.494,0.8}{#1}}%
\newcommand{\hlcom}[1]{\textcolor[rgb]{0.678,0.584,0.686}{\textit{#1}}}%
\newcommand{\hlopt}[1]{\textcolor[rgb]{0,0,0}{#1}}%
\newcommand{\hlstd}[1]{\textcolor[rgb]{0.345,0.345,0.345}{#1}}%
\newcommand{\hlkwa}[1]{\textcolor[rgb]{0.161,0.373,0.58}{\textbf{#1}}}%
\newcommand{\hlkwb}[1]{\textcolor[rgb]{0.69,0.353,0.396}{#1}}%
\newcommand{\hlkwc}[1]{\textcolor[rgb]{0.333,0.667,0.333}{#1}}%
\newcommand{\hlkwd}[1]{\textcolor[rgb]{0.737,0.353,0.396}{\textbf{#1}}}%
\let\hlipl\hlkwb

\usepackage{framed}
\makeatletter
\newenvironment{kframe}{%
 \def\at@end@of@kframe{}%
 \ifinner\ifhmode%
  \def\at@end@of@kframe{\end{minipage}}%
  \begin{minipage}{\columnwidth}%
 \fi\fi%
 \def\FrameCommand##1{\hskip\@totalleftmargin \hskip-\fboxsep
 \colorbox{shadecolor}{##1}\hskip-\fboxsep
     % There is no \\@totalrightmargin, so:
     \hskip-\linewidth \hskip-\@totalleftmargin \hskip\columnwidth}%
 \MakeFramed {\advance\hsize-\width
   \@totalleftmargin\z@ \linewidth\hsize
   \@setminipage}}%
 {\par\unskip\endMakeFramed%
 \at@end@of@kframe}
\makeatother

\definecolor{shadecolor}{rgb}{.97, .97, .97}
\definecolor{messagecolor}{rgb}{0, 0, 0}
\definecolor{warningcolor}{rgb}{1, 0, 1}
\definecolor{errorcolor}{rgb}{1, 0, 0}
\newenvironment{knitrout}{}{} % an empty environment to be redefined in TeX

\usepackage{alltt}
\usepackage[sidebar=false,showpagetotal=false]{../../vimdocs/Vorlagen/beamerthemeSTAT}
\usepackage[sidebar=false,showpagetotal=false]{../../vimdocs/Vorlagen/beamercolorthemeSTAT}
\usepackage{../../vimdocs/Vorlagen/beamerinnerthemeSTAT}
\usepackage{../../vimdocs/Vorlagen/beamerouterthemeSTAT}
\usepackage{../../vimdocs/Vorlagen/beamerfontthemeSTAT}
%\usepackage{../../vimdocs/Vorlagen/Sweave} % disable Sweave when using knitr
\usepackage{listings}
\usepackage{multicol}
% bibliography
%\usepackage[round]{natbib}
%\bibliography{lib}

\bibpunct{(}{)}{,}{a}{}{,}
%\usepackage[ngerman]{babel} 


\hypersetup{colorlinks=true, breaklinks=true, linkcolor=STATred, menucolor=STATred, urlcolor=STATred}
\newcommand{\Rlogo}{\raisebox{-.1mm}{\includegraphics[width=1.2em]{../../vimdocs/Vorlagen/Rlogo.pdf}}}
\newcommand{\mymat}[1]{\boldsymbol{#1}}
\newcommand{\m}[1]{\boldsymbol{#1}}
\newcommand{\R}{\textsf{R}}
\newcommand{\code}[1]{\texttt{#1}}     
\bibpunct{(}{)}{,}{a}{}{,}
\lstloadlanguages{R} 
 \lstdefinelanguage{Renhanced}[]{R},% 
   alsoother={:_\$}} 
 \lstset{language=Renhanced,extendedchars=true, 
   basicstyle=\small\ttfamily, 
   commentstyle=\color{STATgrey}, 
   keywordstyle=\color{STATred}\bfseries,
   showstringspaces=false, 
   index=[1][keywords], 
   indexstyle=\indexfonction
} 
\newcommand{\indexfonction}[1]{\index{#1@\texttt{#1}}}
\newcommand{\boma}[1]{\mbox{\boldmath ${#1}$}}
\newcommand{\MD}{\mbox{MD}}
\newcommand{\Cov}{\mbox{Cov}}
\newcommand{\argmax}{\mathop{\mbox{argmax}}}
\newcommand{\argmin}{\mathop{\mbox{argmin}}}
\newcommand{\ab}{\boldsymbol{a}}
\newcommand{\betab}{\boldsymbol{\beta}}
\newcommand{\med}{\mathop{\mbox{median}}}
\newcommand{\mad}{\mathop{\mbox{MAD}}}
\newcommand{\wb}{\boldsymbol{w}}

\definecolor{Gray}{gray}{0.75}
\definecolor{pddblue}{rgb}{.17,.31,.44}
% Farben wie fuer TU
\definecolor{pddblue}{rgb}{.17,.31,.44}
\definecolor{pddblue2}{rgb}{.17,.36,.49}
\definecolor{pddblue3}{rgb}{.2,.35,.60}
\definecolor{tublau}{rgb}{0,0.4,0.66} % ca. #0066AA
\definecolor{rotgold}{rgb}{0.8,0.3,0}
\definecolor{pdlblue}{rgb}{.75,.85,.92}
\definecolor{pdllblue}{rgb}{.9,.95,.98}
\definecolor{pdlparchment}{rgb}{.96,.94,.89}
\definecolor{pddparchment}{rgb}{.89,.85,.69}
\definecolor{Gray}{gray}{0.75}
\definecolor{Gray2}{gray}{0.5}
\definecolor{digreen}{rgb}{.1,.31,.1}
\definecolor{diblue}{rgb}{.0,.2,.35}
\definecolor{diblue2}{rgb}{.0,.3,.2}


% ---------------------
% begin of presentation
% ---------------------
\IfFileExists{upquote.sty}{\usepackage{upquote}}{}
\begin{document}
% \SweaveOpts{concordance=TRUE} % disable Sweave when using knitr
% define title and authors
\title{Implementation of the Cell-Key Method \& Targeted Record Swapping}

\author[Gussenbauer, Kowarik, Meindl]{\textbf{Johannes Gussenbauer}, Alexander Kowarik, Bernhard Meindl}
\institute[www.statistik.at]{Statistik Austria}
\date[November 2018]{November, 2018}

% slides
\pgfdeclareimage[height=0.87cm,width=2cm]{logo}{../../vimdocs/Vorlagen/logo}
\logo{\pgfuseimage{logo}}
{
\usebackgroundtemplate{\includegraphics[height=0.95\paperheight,width=\paperwidth]{../../vimdocs/Vorlagen/titlepage}}
\begin{frame}
\titlepage
\end{frame}
}
\setcounter{tocdepth}{1}

\begin{knitrout}
\definecolor{shadecolor}{rgb}{0.969, 0.969, 0.969}\color{fgcolor}\begin{kframe}
\begin{alltt}
\hlstd{knitr}\hlopt{::}\hlstd{opts_chunk}\hlopt{$}\hlkwd{set}\hlstd{(}\hlkwc{size}\hlstd{=}\hlstr{"footnotesize"}\hlstd{)}
\end{alltt}
\end{kframe}
\end{knitrout}


% INTRODUCTION
\begin{frame}[fragile]{Introduction}
\begin{itemize}
  \item Cell-Key Method and Targeted Record Swapping implemented in R-Packages
  \item Available on https://github.com/sdcTools
  \begin{itemize}
    \item \texttt{recordSwapping}
    \item \texttt{cellKey}
  \end{itemize}
  \item Implementations are prototype-ready
\end{itemize}
\end{frame}

% CELL KEY
\begin{frame}[fragile]{cellKey}
\begin{itemize}
  \item Two different approaches implemented
  \begin{itemize}
    \item \textit{Methodology for the Automatic Confidentialisation of Statistical Outputs from Remote Servers at the Australian Bureau of Statistics} (Thompson, Broadfoot, Elazar) 
    \item Approach developed by the Federal Statistical Office of Germany (Destatis)
  \end{itemize}
  \item \texttt{cellKey} depends on R-package \texttt{ptable}
\end{itemize}
\end{frame}

% CELL KEY
\begin{frame}[fragile]{Main Features}
\begin{itemize}
  \item Methods \texttt{abs} and \texttt{destatis}
  \item \texttt{ck\_generate\_rkeys()} for generating record keys
  
  \begin{itemize}
    \item perturbation for magnitude tables only for \texttt{abs} 
  \end{itemize}
  
  \item main function \texttt{pertubTable()}
  \item allows sampling weights
  \item define arbitrarily complex hierarchies like in \texttt{sdcTable}
  
  \begin{itemize}
    \item improved functionality in \texttt{cellKey}
  \end{itemize}
  
  \item various auxiliary methods implemented    
\end{itemize}
\end{frame}

% CELL KEY
\begin{frame}[fragile]{Example}
\begin{knitrout}\footnotesize
\definecolor{shadecolor}{rgb}{0.969, 0.969, 0.969}\color{fgcolor}\begin{kframe}
\begin{alltt}
\hlcom{# load package}
\hlkwd{library}\hlstd{(cellKey,}\hlkwc{verbose}\hlstd{=}\hlnum{FALSE}\hlstd{)}
\end{alltt}


{\ttfamily\noindent\itshape\color{messagecolor}{\#\# Loading required package: data.table}}\begin{alltt}
\hlcom{# load dummy data}
\hlstd{dat} \hlkwb{<-} \hlkwd{ck_create_testdata}\hlstd{()[,}\hlkwd{c}\hlstd{(}\hlstr{"sex"}\hlstd{,}\hlstr{"age"}\hlstd{,}\hlstr{"savings"}\hlstd{,}
                               \hlstr{"income"}\hlstd{,}\hlstr{"sampling_weight"}\hlstd{)]}
\hlstd{dat[,cnt_males}\hlkwb{:=}\hlkwd{ifelse}\hlstd{(sex}\hlopt{==}\hlstr{"male"}\hlstd{,} \hlnum{1}\hlstd{,} \hlnum{0}\hlstd{)]}
\hlstd{dat[,cnt_highincome}\hlkwb{:=}\hlkwd{ifelse}\hlstd{(income}\hlopt{>=}\hlnum{9000}\hlstd{,} \hlnum{1}\hlstd{,} \hlnum{0}\hlstd{)]}
\end{alltt}
\end{kframe}
\end{knitrout}
$\rightarrow$ create a perturbed table of counts of variables \texttt{sex} by \texttt{age} 
\end{frame}

% CELL KEY
\begin{frame}[fragile]{Set parameters}

\begin{knitrout}\footnotesize
\definecolor{shadecolor}{rgb}{0.969, 0.969, 0.969}\color{fgcolor}\begin{kframe}
\begin{alltt}
\hlstd{pert_params} \hlkwb{<-} \hlkwd{ck_create_pert_params}\hlstd{(}
  \hlkwc{bigN}\hlstd{=}\hlnum{17312941}\hlstd{,}
  \hlkwc{smallN}\hlstd{=}\hlnum{12}\hlstd{,}
  \hlkwc{pTable}\hlstd{=}\hlkwd{ck_create_pTable}\hlstd{(}\hlkwc{D}\hlstd{=}\hlnum{5}\hlstd{,} \hlkwc{V}\hlstd{=}\hlnum{3}\hlstd{,} \hlkwc{pTableSize}\hlstd{=}\hlnum{70}\hlstd{,} \hlkwc{type}\hlstd{=}\hlstr{"abs"}\hlstd{),}
  \hlkwc{sTable}\hlstd{=}\hlkwd{ck_generate_sTable}\hlstd{(}\hlkwc{smallC}\hlstd{=}\hlnum{12}\hlstd{),}
  \hlkwc{mTable}\hlstd{=}\hlkwd{c}\hlstd{(}\hlnum{0.6}\hlstd{,}\hlnum{0.4}\hlstd{,}\hlnum{0.2}\hlstd{))}
\end{alltt}
\end{kframe}
\end{knitrout}
\end{frame}

% CELL KEY
\begin{frame}[fragile]{Create input}

\begin{knitrout}\footnotesize
\definecolor{shadecolor}{rgb}{0.969, 0.969, 0.969}\color{fgcolor}\begin{kframe}
\begin{alltt}
\hlstd{inp} \hlkwb{<-} \hlkwd{ck_create_input}\hlstd{(}
  \hlkwc{dat}\hlstd{=dat,}
  \hlkwc{def_rkey}\hlstd{=}\hlnum{15}\hlopt{*}\hlkwd{nrow}\hlstd{(dat),}
  \hlkwc{pert_params}\hlstd{=pert_params)}
\hlkwd{print}\hlstd{(}\hlkwd{class}\hlstd{(inp))}
\end{alltt}
\begin{verbatim}
## [1] "pert_inputdat"
## attr(,"package")
## [1] "cellKey"
\end{verbatim}
\end{kframe}
\end{knitrout}
\end{frame}

% CELL KEY
\begin{frame}[fragile]{Specify Dimensions}

\begin{knitrout}\footnotesize
\definecolor{shadecolor}{rgb}{0.969, 0.969, 0.969}\color{fgcolor}\begin{kframe}
\begin{alltt}
\hlcom{# example for variable sex}
\hlstd{dim.sex} \hlkwb{<-} \hlkwd{data.table}\hlstd{(}\hlkwc{levels}\hlstd{=}\hlkwd{c}\hlstd{(}\hlstr{"@"}\hlstd{,}\hlstr{"@@"}\hlstd{,}\hlstr{"@@"}\hlstd{),}
                      \hlkwc{codes}\hlstd{=}\hlkwd{c}\hlstd{(}\hlstr{"Total"}\hlstd{,} \hlstr{"male"}\hlstd{,} \hlstr{"female"}\hlstd{))}
\hlkwd{print}\hlstd{(dim.sex)}
\end{alltt}
\begin{verbatim}
##    levels  codes
## 1:      @  Total
## 2:     @@   male
## 3:     @@ female
\end{verbatim}
\end{kframe}
\end{knitrout}

\begin{knitrout}\footnotesize
\definecolor{shadecolor}{rgb}{0.969, 0.969, 0.969}\color{fgcolor}\begin{kframe}
\begin{alltt}
\hlcom{# or alternatively}
\hlstd{dim.sex2} \hlkwb{<-} \hlkwd{ck_create_node}\hlstd{(}\hlkwc{total_lab}\hlstd{=}\hlstr{"Total"}\hlstd{)}
\hlstd{dim.sex2} \hlkwb{<-} \hlkwd{ck_add_nodes}\hlstd{(dim.sex2,} \hlkwc{reference_node}\hlstd{=}\hlstr{"Total"}\hlstd{,}
                         \hlkwc{node_labs}\hlstd{=}\hlkwd{c}\hlstd{(}\hlstr{"male"}\hlstd{,}\hlstr{"female"}\hlstd{))}
\hlkwd{print}\hlstd{(dim.sex2)}
\end{alltt}
\begin{verbatim}
##    levelName
## 1 Total     
## 2  ¦--male  
## 3  °--female
\end{verbatim}
\end{kframe}
\end{knitrout}
\end{frame}

% CELL KEY
\begin{frame}[fragile]{Specify Dimensions}

\begin{knitrout}\footnotesize
\definecolor{shadecolor}{rgb}{0.969, 0.969, 0.969}\color{fgcolor}\begin{kframe}
\begin{alltt}
\hlcom{# example for variable age}
\hlstd{dim.age} \hlkwb{<-} \hlkwd{data.table}\hlstd{(}\hlkwc{levels}\hlstd{=}\hlkwd{c}\hlstd{(}\hlstr{"@"}\hlstd{,}\hlkwd{rep}\hlstd{(}\hlstr{"@@"}\hlstd{,} \hlnum{6}\hlstd{)),}
                      \hlkwc{codes}\hlstd{=}\hlkwd{c}\hlstd{(}\hlstr{"Total"}\hlstd{,} \hlkwd{paste0}\hlstd{(}\hlstr{"age_group"}\hlstd{,}\hlnum{1}\hlopt{:}\hlnum{6}\hlstd{)))}

\hlcom{# or alternatively}
\hlstd{dim.age2} \hlkwb{<-} \hlkwd{ck_create_node}\hlstd{(}\hlkwc{total_lab}\hlstd{=}\hlstr{"Total"}\hlstd{)}
\hlstd{dim.age2} \hlkwb{<-} \hlkwd{ck_add_nodes}\hlstd{(dim.age2,} \hlkwc{reference_node}\hlstd{=}\hlstr{"Total"}\hlstd{,}
                         \hlkwc{node_labs}\hlstd{=}\hlkwd{paste0}\hlstd{(}\hlstr{"age_group"}\hlstd{,}\hlnum{1}\hlopt{:}\hlnum{6}\hlstd{))}
\hlkwd{print}\hlstd{(dim.age2)}
\end{alltt}
\begin{verbatim}
##        levelName
## 1 Total         
## 2  ¦--age_group1
## 3  ¦--age_group2
## 4  ¦--age_group3
## 5  ¦--age_group4
## 6  ¦--age_group5
## 7  °--age_group6
\end{verbatim}
\end{kframe}
\end{knitrout}
\end{frame}

% CELL KEY
\begin{frame}[fragile]{Compine Hierachies}
\begin{knitrout}\footnotesize
\definecolor{shadecolor}{rgb}{0.969, 0.969, 0.969}\color{fgcolor}\begin{kframe}
\begin{alltt}
\hlstd{dimList} \hlkwb{<-} \hlkwd{list}\hlstd{(}\hlkwc{sex}\hlstd{=dim.sex,} \hlkwc{age}\hlstd{=dim.age2)}
\hlkwd{print}\hlstd{(dimList)}
\end{alltt}
\begin{verbatim}
## $sex
##    levels  codes
## 1:      @  Total
## 2:     @@   male
## 3:     @@ female
## 
## $age
##        levelName
## 1 Total         
## 2  ¦--age_group1
## 3  ¦--age_group2
## 4  ¦--age_group3
## 5  ¦--age_group4
## 6  ¦--age_group5
## 7  °--age_group6
\end{verbatim}
\end{kframe}
\end{knitrout}
\end{frame}

\begin{frame}[fragile]{Perturb Table}
\begin{knitrout}\footnotesize
\definecolor{shadecolor}{rgb}{0.969, 0.969, 0.969}\color{fgcolor}\begin{kframe}
\begin{alltt}
\hlstd{tab1} \hlkwb{<-} \hlkwd{perturbTable}\hlstd{(}\hlkwc{inp}\hlstd{=inp,} \hlkwc{dimList}\hlstd{=dimList,}
                     \hlkwc{countVars}\hlstd{=}\hlkwd{c}\hlstd{(}\hlstr{"cnt_males"}\hlstd{,} \hlstr{"cnt_highincome"}\hlstd{),}
  \hlkwc{weightVar}\hlstd{=}\hlstr{"sampling_weight"}\hlstd{,} \hlkwc{numVars}\hlstd{=}\hlkwa{NULL}\hlstd{)}
\hlkwd{print}\hlstd{(tab1)}
\end{alltt}
\begin{verbatim}
## The weighted 2-dimensional table consists of 21 cells. The results are based on all units in the input data. 
## The dimensions are given by the following variables
## o sex
## o age 
## 
## Type of pTable-used: 'abs' 
## The following count-variables have been tabulated/perturbed:
## o Total
## o cnt_males
## o cnt_highincome 
## No numeric variables have been tabulated/perturbed in this table
\end{verbatim}
\end{kframe}
\end{knitrout}
\end{frame}

\begin{frame}[fragile]{Perturbed Table}
  \begin{itemize}
    \item returns tables with \texttt{ck\_freq\_table()} 
\begin{knitrout}\footnotesize
\definecolor{shadecolor}{rgb}{0.969, 0.969, 0.969}\color{fgcolor}\begin{kframe}
\begin{alltt}
 \hlcom{# count table containing}
 \hlcom{# original, perturbed and (un)weighted values}
 \hlkwd{ck_freq_table}\hlstd{(tab1,} \hlkwc{vname}\hlstd{=}\hlstr{"cnt_males"}\hlstd{)}
\end{alltt}
\end{kframe}
\end{knitrout}

    \item compute information loss measures with \texttt{ck\_cnt\_measures()}
\begin{knitrout}\footnotesize
\definecolor{shadecolor}{rgb}{0.969, 0.969, 0.969}\color{fgcolor}\begin{kframe}
\begin{alltt}
\hlkwd{ck_cnt_measures}\hlstd{(tab1,} \hlkwc{vname}\hlstd{=}\hlstr{"Total"}\hlstd{)}
\end{alltt}
\end{kframe}
\end{knitrout}
  \end{itemize}
\end{frame}



\begin{frame}[fragile]{Perturbed Table}
\begin{itemize}
  \item perturbed table of continous data
\end{itemize}
\begin{knitrout}\footnotesize
\definecolor{shadecolor}{rgb}{0.969, 0.969, 0.969}\color{fgcolor}\begin{kframe}
\begin{alltt}
\hlstd{tab2} \hlkwb{<-} \hlkwd{perturbTable}\hlstd{(}\hlkwc{inp}\hlstd{=inp,}\hlkwc{dimList}\hlstd{=dimList,}\hlkwc{weightVar}\hlstd{=}\hlstr{"sampling_weight"}\hlstd{,}
  \hlkwc{countVars}\hlstd{=}\hlkwd{c}\hlstd{(}\hlstr{"cnt_males"}\hlstd{,} \hlstr{"cnt_highincome"}\hlstd{),}
  \hlkwc{numVars}\hlstd{=}\hlkwd{c}\hlstd{(}\hlstr{"savings"}\hlstd{,}\hlstr{"income"}\hlstd{))}

\hlstd{p_income} \hlkwb{<-} \hlkwd{ck_cont_table}\hlstd{(tab2,} \hlkwc{vname}\hlstd{=}\hlstr{"income"}\hlstd{,} \hlkwc{meanBeforeSum}\hlstd{=}\hlnum{TRUE}\hlstd{)}
\hlkwd{head}\hlstd{(p_income)}
\end{alltt}
\begin{verbatim}
##      sex        age UW_income pUW_income  WS_income pWS_income pWM_income
## 1: Total      Total  22952978 22930849.1 1363517148 1362202587   5006.736
## 2: Total age_group1   9810547  9794206.1  583047435  582076283   4966.648
## 3: Total age_group2   5692119  5679273.0  336517278  335757827   4968.743
## 4: Total age_group3   4406946  4398044.8  261341079  260813219   5090.330
## 5: Total age_group4   2133543  2133021.2  128441306  128409893   5054.513
## 6: Total age_group5    848151   857505.6   49990424   50541786   5073.967
\end{verbatim}
\end{kframe}
\end{knitrout}
\end{frame}

\begin{frame}[fragile]{Perturbed Table}
\begin{itemize}
  \item perturbed table for a specific group $\rightarrow$ \texttt{by="cnt\_highincome"}
\end{itemize}
\begin{knitrout}\footnotesize
\definecolor{shadecolor}{rgb}{0.969, 0.969, 0.969}\color{fgcolor}\begin{kframe}
\begin{alltt}
\hlstd{tab3} \hlkwb{<-} \hlkwd{perturbTable}\hlstd{(}\hlkwc{inp}\hlstd{=inp,} \hlkwc{dimList}\hlstd{=dimList,}
  \hlkwc{weightVar}\hlstd{=}\hlstr{"sampling_weight"}\hlstd{,}
  \hlkwc{numVars}\hlstd{=}\hlkwd{c}\hlstd{(}\hlstr{"savings"}\hlstd{),} \hlkwc{by}\hlstd{=}\hlstr{"cnt_highincome"}\hlstd{)}
\end{alltt}
\end{kframe}
\end{knitrout}
\end{frame}

\end{document}

